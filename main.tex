\documentclass{article}

% Language setting
% Replace `english' with e.g. `spanish' to change the document language
\usepackage[german]{babel}

% Set page size anad margins
% Replace `letterpaper' with`a4paper' for UK/EU standard size
\usepackage[letterpaper,top=2cm,bottom=2cm,left=3cm,right=3cm,marginparwidth=1.75cm]{geometry}

% Useful packages
\usepackage{amsmath}
\usepackage{graphicx}
\usepackage[colorlinks=true, allcolors=blue]{hyperref}

\title{Ethical Evaluation of CoEnv}
\author{Max Richter, and ...}

\begin{document}
\maketitle

\begin{abstract}
Das Ziel dieses Textes ist es, unser Produkt CoEnv unter ethischen Gesichtspunkten zu bewerten. CoEnv ist ein modulares Ökosystem aus kleineren Produkten welche das Zusammenarbeiten in geteilten Umgebungen erleichtern sollen.
\end{abstract}

\section{Einleitung}

CoEnv ist ein im Rahmen des Kurses "Function Objects for Shared Environments" entstandenes Konzept. 

Diese Produkte sollen gemeinsam das kommunizieren von Bedürfnissen und 

Es besteht im Moment aus drei einzelnen Objekten.

\subsubsection*{RoomBoard} 

Das RoomBoard ist eine interaktive Schaltfläche...

\subsubsection*{Cube}

Der Cube ist eine kleine low-power Arbeitslampe ...

\subsubsection*{Siegel}

Die Siegel sind kleine Symbole...

\subsection{Innovation von CoEnv}

tbd.

\subsection{Chancen die durch CoEnv entstehen}

tbd.

\section{Akteur Analyse}

\subsection{Welche Akteure gibt es?}

tbd.

\subsection{Welche Ziele haben die Akteure?}

tbd.

\subsection{Mit welchen Mitteln sollen diese Ziele erreicht werden?}

tbd.

\section{Folgenanalyse}

\subsection{Positive Folgen die durch CoEnv entstehen}

tbd.

\section{Zusatzarbeit...}

\end{document} 
