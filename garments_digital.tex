\documentclass{article}

% Language setting
% Replace `english' with e.g. `spanish' to change the document language
\usepackage[german]{babel}
\newcommand{\comment}[1]{}
% Set page size and margins
% Replace `letterpaper' with`a4paper' for UK/EU standard size
\usepackage[letterpaper,top=2cm,bottom=2cm,left=3cm,right=3cm,marginparwidth=1.75cm]{geometry}
\usepackage[nottoc]{tocbibind}
\usepackage{csquotes}
\MakeOuterQuote{"}
% Useful packages
\usepackage{amsmath}
\usepackage[colorlinks=true, allcolors=blue]{hyperref}
\usepackage{graphicx}
\usepackage[colorlinks=true, allcolors=blue]{hyperref}
\usepackage{fancyhdr}

\pagestyle{fancy}
\fancyhf{}
\lhead{Risiko fokussierte ethische Bewertung von Garments Digital}
\rhead{Kay Schuh, Max Richter}
\rfoot{Seite \thepage}

\title{Risiko fokussierte ethische Bewertung von Garments Digital}
\author{Kay Schuh, Max Richter}

\begin{document}
\maketitle

\begin{abstract}
In diesem Text werden wir die negativen ethischen Aspekte von Garments Digital erläutern.
\end{abstract}

\newpage
\tableofcontents

\newpage

\section{Akteur Analyse}

\subsection{Selbstständige Designer*innen}
\subsubsection{Interessen}
Privatpersonen mit Interesse und Kenntnis in Sachen Gestaltung wollen sich entfalten, kreativ sein und daran etwas verdienen.
\subsubsection{Ziele}
Die Designer wollen mit erstellten Designs Nutzer ansprechen, einen Anteil an den Verkäufen erhalten und ihre Reichweite und resultierend den eigenen Gewinn steigern. Auch ist es im Interesse der Designer, sich durch die Automatisierung und Vorzüge der Software Aufwand zu ersparen.
\subsubsection{Mittel}
Garments Digital bietet den designenden Nutzern ein Tool und eine Plattform zum Erstellen und der Präsentation der eigenen Werke sowie dem vereinfachten Austausch mit Kunden. Der Vertrieb wird für die Designer übernommen und sie sind auch in der Lage, NFTs zu erstellen und als digitale Artefakte zu verkaufen.
\subsubsection{Werte}
Anerkennung, Erwerb, Anspruch auf Urheberrecht, Einfachheit

\subsection{Fashion Retailer}
\subsubsection{Interessen}
Die meisten Retailer sind kommerzielle Unternehmen und somit ist eines ihrer Hauptinteressen wirtschaftliches Handeln, oder genauer gesagt, Profit machen.
\subsubsection{Ziele}
Retailer machen Profit, indem sie Klamotten billig einkaufen und diese dann teurer verkaufen. Daraus ergeben sich zwei Ziele, zum einen, dass sie Klamotten weiter günstig einkaufen können und zum anderen, dass es einen Absatzmarkt und Interesse für ihre Kleidung gibt.
\subsubsection{Mittel}
Damit sie die Kleidung günstig einkaufen können, bedienen sie sich an der freien Marktwirtschaft und damit dem Konkurrenzdruck der Kleidungsproduzenten untereinander und dem Fakt, dass Arbeitsschutzgesetze in vielen ärmeren Ländern nicht ausgeprägt oder nicht vorhanden sind. Um ihre Kleidung wieder zu verkaufen, versuchen sie die Kaufentscheidung der Konsumenten durch Marketing zu beeinflussen. Außerdem versuchen größere Konzerne durch Lobbyarbeit politische Entscheidungen zu beeinflussen \cite{LobbyFactsHM}
\subsubsection{Werte}
n.A.

\subsection{Endverbraucher}
\subsubsection{Interessen}
Der Endverbraucher ist der Käufer der designten Kleidungsstücke. Er wünscht sich Qualität zu einem guten Preis und speziell in diesem Fall ein trendiges und individuelles Produkt.
\subsubsection{Ziele}
Käufer das Produkt, das sie sich vorstellen, und auf einfachem Weg an dieses gelangen. Die Endverbraucher von Garments Digital mögen Interesse an einem speziellen Design mit bestimmter Passform haben, welches ihren Vorstellungen von Preis und Qualität entspricht.
\subsubsection{Mittel}
Durch die Einfachheit des Designs und der Vorstellung der Produkte in einer Art digitalem Showroom ist das Erstellen und Wählen eines Produktes vereinfacht und somit günstiger. Die Wahrscheinlichkeit einer Reklamation und eines unzufriedenen Endverbrauchers wird verringert. Der Weg zum Produkt wird durch die digitale Plattform und das Tool verkürzt und erleichtert.
\subsubsection{Werte}
Einfachheit, Individualisierung, Sicherheit, Qualität

\newpage

\section{Negative Folgenanalyse}

\subsection{Gruppen/Markenzwang}
Soziale Medien üben einen großen Druck auf ihrer User*innen aus, gerade Mode, Beauty und Lifestyle orientierte Plattformen wie Instagram. Wo früher darauf geachtet wurde, wer im Freundeskreis welche Markenkleidung trägt, kann dies heute von Tausenden von Menschen in Echtzeit getan werden. Dabei unterstützen soziale Medien die "Fear of missing Out" indem sie es User*innen ermöglichen sofort und überall up-to-date zu sein. Diese "Fear of missing out" ist mit einer geringeren Bedürfnis-, Stimmungs- und Lebenszufriedenheit verbunden \cite{Przybylski2013}. 
\\[2ex]
Digitale Kleidung im Allgemeinen könnte diesen Effekt von FoMO in Sozialen Medien verstärken, indem das Kaufen und Sharen der neuesten Kollektionen nur einen Knopfdruck entfernt sind.
\\[2ex]
Ein weiterer Effekt, den man jetzt schon bei Spielen wie zum Beispiel Fortnite beobachten kann ist, dass sich Kinder, da sie noch keinen Zugang zu digitalen Zahlungsmitteln haben, die Kreditkarten der Eltern nehmen und damit bezahlen.

\subsection{Gefahren der digitalen Anonymität}
Mit den Chancen der Anonymität in sozialen Medien kommen auch einige Risiken. So sind Menschen eher zu virtueller Gewalt bereit wenn sie sich anonym fühlen \cite{1971-08069-001}.
\\[2ex]
Durch die Demokratisierung des Erstellens von Modedesigns und die Anonymität des Internets könnte es verstärkt zu gruppenbezogener Menschenfeindlichkeit in Form von Kleidung kommen. Wo Personen in der realen Welt mit direkten Konsequenzen zu rechnen haben, wenn sie zum Beispiel rassistische Motive auf der Kleidung tragen, ist dies in der virtuellen Welt nicht unbedingt gegeben.

\subsection{Verlust von Arbeitsplätzen}
Dieser Effekt ist mit den meisten Neuerungen verbunden, die ein schon vorhandenes Gewerbe digitalisieren.

\subsection{Verlust eines Handwerks}
Wenn man davon ausgeht, dass vor allem in bestimmten Bereichen der Mode digitale Kleidung die neue Norm wird, könnte dies dazu führen, dass die handwerklichen Tätigkeiten, die mit diesen Bereichen verbunden sind, nicht weiter gebraucht werden. Dies wiederum könnte zu einem Verlust von Wissen und Können führen.

\subsection{NFT's und das Klima}
Wenn Garments Digital es erlauben wird sehr einfach eigene Designs zu "minten", dass heißt zum Verkauf auf NFT Plattformen anzubieten, würde das diese NFT Plattformen unterstützen. Ein Problem was viele NFT Plattformen noch haben ist, dass ihre Technik auf einem Algorithmus namens "Proof of Work" aufbaut, dessen Ziel es ist Besitzansprüche durch Rechenleistung zu klären. Wenn man davon ausgeht, dass nicht aller Strom regenerativ sein wird, könnte dies zu einer weiteren Belastung des Weltklimas führen.


\newpage

\comment{

QUELLEN:

Kommunikation durch Mode -
Entwicklungen und Veränderungen
http://othes.univie.ac.at/23661/1/2012-09-22_0404734.pdf

Zu den Funktionen der Mode
https://unipub.uni-graz.at/lithes/content/titleinfo/1678010/full.pdf

Gefahren sozialer Medien für Jugendliche
https://www.jugendundmedien.ch/themen/selbstdarstellung-und-schoenheitsideale

Wie Social-Media-Idole das Selbstbild von Jugendlichen prägen
"Bereits Erstklässlerinnen machen sich Gedanken über ihre Figur. Sie vergleichen den Umfang ihrer Oberschenkel mit den Körpermassen ihrer Schulkameradinnen und diskutieren darüber, ob sie an Gewicht verlieren sollten."

Identitätsbildung in Sozialen Medien
https://link.springer.com/referenceworkentry/10.1007%2F978-3-658-03895-3_4-2
https://link.springer.com/referenceworkentry/10.1007%2F978-3-658-03895-3_4-2

Fomo and Social Media
https://www.sciencedirect.com/science/article/pii/S0747563213000800

H und M Group Lobby register
https://lobbyfacts.eu/representative/3ecbf1db240a4075bea44b5bdb13c76c/h-m-hennes-mauritz-ab

The positive and negative implications of anonymity in Internet social interactions: “On the Internet, Nobody Knows You’re a Dog”
https://www.sciencedirect.com/science/article/pii/S0747563206001221?via%3Dihub

IDEEN:

- Stärkeren Druck auf Teenager, die neuesten digitale Mode zu haben, so e.g. Fortnite Skins und Mamas Kreditkarte ziehen
    - Influencer üben stärkeren Druck aus? Digitale Kleidung könnte das unterstützen
    - Influencer sind eher materialistisch
- Anonymität, könnte zu Hakenkreuz Pullis führen
- Retailer / herkömmliche Print Shops verlieren Kundschaft, da ihre Methoden nicht so innovativ, einfach und günstig sind
- Auch für die Produktfotografie entfallen Arbeit und Jobs

TODOS:

Erstellt eine negative Sach- bzw. Folgenanalyse 
- basierend auf der vorhandenen Beschreibung des Themenfelds und der positiven Sach- bzw. Folgenanalysen aus Task 2 mit einer negativen Sachanalyse, je Akteur (bereits erwahnt, zusätzlich : 

- [ ] Welche Interessw haben diese? 
- [ ] Welche Ziele verfolgen sie mit diesen Interessen? 
- [ ] Mit welchen Mitteln sollen diese Ziele erreicht werden? 
- [ ] Welche Werte stecken dahinter? 

- [ ] Einer negativen Folgenanalyse 
    - Welche negativen Folgen entstehen aus der neuen Technologie? 
    - Beschreibt die Folgen anhand der Vorlesungsstruktur (Perspektiven-gebunden, Auftretenszeit-NVahrscheinlichkeitsgebunden) 
    - Stellt einen negativen Wertebezug her und erhebt einen negativen Normativitatsanspruch (So darf es nicht kommen) 
    
Ohne Analyse der resultierenden Wertekonflikte 

}

\bibliographystyle{unsrt}
\bibliography{sample}

\end{document}t
